% sections/01-introduction.tex

\section{Introduction}

The smart contract has become a predominant method for developing decentralized applications (dApps), and the Ethereum Virtual Machine (EVM) is widely integrated into various blockchains. Despite notable advancements in scalability within the blockchain realm, smart contracts, especially EVM-based ones, are still limited in terms of fully realizing dApp functionality due to the restricted extensibility of the underlying layer. Aspect Programming aims to address this limitation, enhancing the programmability of the blockchain and modular blockchain execution layer while ensuring compatibility with existing smart contracts and their ecosystems.

In this paper, we present Aspect Programming, a programming model that enables native extensions on the blockchain. Our discussion of Aspect Programming includes the following sections:

\begin{enumerate}
  \item Offer a brief overview of the issues of blockchain extensibility and present the native extensions as a solution. We explain how native extensions address these issues and discuss their advantages for both developers and users. (Section 2)
  \item Provide an overview of how to implement native extensions with Aspect Programming and introduce the structure and key features of Aspect Programming. (Section 3)
  \item A deep dive into the technical details of Aspect Programming, covering its in-depth model design (Section 4) and implementation (Section 5), supplementing with miscellanea about design thinking (Section 8). It aims to foster a clearer understanding of Aspect Programming and its functionality.
  \item Introduce what innovative application features Aspect brings to dApps, and describe the impact on the broader blockchain ecosystem, including infrastructure, middleware, and application building. (Section 6)
  \item Conclude by summarizing what we have done for Aspect Programming, describing the plans for its ongoing enhancement and adoption, and outlining its possible future features. (Section 7)
\end{enumerate}
